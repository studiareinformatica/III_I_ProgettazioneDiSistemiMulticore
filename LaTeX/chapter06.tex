\section{Legge di Amdahl e Gustafson}

\subsection{Domanda 1}
Sono forniti dua algoritmi per risolvere un problema su input di dimensione $n$: $A_1$ esegue $n^3$ operazioni, con il $40\%$ di tempo di esecuzione seriale; $A_2$ esegue $n^4$ operazioni, con lo $0.1\%$ di tempo di esecuzione seriale.
\begin{enumerate}
	\item Se $A_1$ viene eseguito su una piattaforma con 2 processori e $n = 10^2$, di quanti processori $P$ abbiamo bisogno perch\`{e} l'esecuzione di $A_2$ su $n$ risulti più veloce usando $P$ processori?
	\paragraph{Soluzione}
	$T_P = T(f) + T(s) = f + (1-f)/n$. 	Quindi, con $n = 10^2$: \\
	$T_{P_1} = \frac{40}{10^2}\times n^3 + \frac{\frac{60}{10^2}\times n^3}{2} = \frac{40}{10^2}\times 10^6 + \frac{\frac{60}{10^2}\times 10^6}{2} = 40\times 10^4 + 30\times 10^4 = 70\times 10^4 = 7\times 10^5$ .\\
	$T_{P_2} = \frac{0,1}{10^2}\times n^4 + \frac{\frac{99,9}{10^2}\times n^4}{P} = \frac{1}{10^3}\times 10^8 +  \frac{\frac{999}{10^3}\times 10^8}{P} = 10^5 + \frac{999}{P}\times 10^5 = 10^5\times(1 + \frac{999}{P})$ . \\
	$T_{P_2} \leq T_{P_1} \Rightarrow 10^5\times(1 + \frac{999}{P}) \leq 7\times 10^5 \Rightarrow 1 + \frac{999}{P} \leq 7 \Rightarrow \frac{999}{P} \leq 6 \Rightarrow \frac{1}{P} \leq \frac{6}{999} \Rightarrow P \geq \frac{333}{2}$ .
	\item Qual è lo speedup di $A_2$ su 100 processori in base alla legge di \textit{Amdahl}?
	\paragraph{Soluzione}
	$S(n) \leq \frac{1}{f + \frac{1-f}{n}}$. Quindi: \\
	$S(100) = \frac{1}{\frac{1}{100}+ \frac{1 - \frac{1}{100}}{100}} = \frac{100}{1 + 1 - \frac{1}{100}} = \frac{100}{\frac{100 + 100 - 1}{100}} = \frac{10^4}{199} \simeq 50$ .
	\item Per lo speedup, è importante che il tempo di esecuzione di $A_2$ sia $n^4$?
	\paragraph{Soluzione}
	No, perché nella derivazione della legge di \textit{Amdahl} il tempo si semplifica.
\end{enumerate}

\newpage

\subsection{Domanda 2}
Rispondete alle seguenti domande, motivando le risposte:
\begin{enumerate}
	\item Un programma ha efficienza 50\% usando 16 processori. Qual è la percentuale di codice seriale del programma in base alla legge di \textit{Amdahl}?
	\item Se un programma $P_1$ ha uno speedup maggiore di un programma $P_2$ e i due programmi eseguono lo stesso work, è vero che $P_1$ è sempre veloce almeno quanto $P_2$? Se l'affermazione è vera dimostratela, altrimenti fornite un controesempio.
	\item Illustrate la legge di \textit{Gustafson}, dimostrandola e spiegandone le implicazioni.
\end{enumerate}

\newpage

\section{Fork/Join}

\subsection{Domanda 1}
Il seguente codice Java calcola lo speculare di un array:
\begin{lstlisting}
static int[] reverse(int[] array) {
	int[] answer = new int[array.length];
	fjPool.invoke(new Reverse(answer, array, 0, array.length));
	rerutn answer;
}

class Reverse extends RecursiveAction {
	int[] in, out;
	int lo, hi;

	Reverse(int[] o, int[] i, int l, int h) {
		out = o; in = i; lo = l; hi = h;
	}

	public void compute() {
		if (hi == lo+1) {
			out[in.length - lo - 1] = in[lo];
		} else {
			Reverse left  =
				new Reverse(out,in,lo,(hi+lo)/2);
			Reverse right =
				new Reverse(out,in,(hi+lo)/2,hi);
			left.fork();
			right.fork();
			right.join();
		}
	}
}
\end{lstlisting}
\begin{itemize}
	\item Il codice contiene un errore: quale?
	\item Disegnate il DAG di esecuzione per $n = 4$, etichettando i nodi del DAG in modo da individuare il thread associato a ciascun nodo e la porzione di array su cui il thread opera.
	\item Discutete \textit{work} e \textit{span} dell'algoritmo in funzione del numero $n$ di elementi dell'array.
	\item Spiegate in modo sintetico come ottimizzare il codice.
	\item Se $n = 2^{30}$, quanti thread fork-join sono creati dal codice (opportunamento corretto come descritto al punto \textit{a})? Applicando un cut-off sequenziale di 1024, quanti thread fork-join sarebbero creati?
	\item Supponete ora di voler memorizzare l'array speculare \textit{in loco}, ovvero di non avere un array di output ma di voler modificare direttamente l'array di input. Come modifichereste il codice in tale scenario?
\end{itemize}

\newpage

\subsection{Domanda 2}
Descrivete l'algoritmo parallelo per la fusione di due array ordinati ($merge$). Illustrate sinteticamente un esempio di esecuzione e analizzate \textit{work} e \textit{span}, assumendo che i due array in input contengano $n$ elementi ciascuno (è sufficiente spiegare come impostare le equazioni di ricorrenza, senza svolgerle esplicitamente).

\newpage

\subsection{Domanda 3}
Immaginate di avere una classe \textit{Worker1} contenente un metodo pubblico \textit{doTask1()}, una classe \textit{Worker2} contenente un metodo pubblico \textit{doTask2()}, \ldots . Supponete inoltre che la vostra applicazione crei quattro thread $T_1$, \ldots, $T_4$ i cui metodi \textit{run()} contengono il seguente codice:
\begin{itemize}
	\item[$T_1$] \textit{worker1.doTask1()}; \textit{worker2.doTask2()};
	\item[$T_2$] \textit{worker5.doTask5()};
	\item[$T_3$] \textit{worker3.doTask3()};
	\item[$T_4$] \textit{worker4.doTask4()};
\end{itemize}
dove le esecuzioni di \textit{doTask1()}, \textit{doTask3()} e \textit{doTask4()} richiedono 20 unit\`{a} di tempo ciascuna, \textit{doTask2()} ne richiede 30 e \textit{doTask5()} 40. Infine, assumente che il thread \textit{main} esegua il seguente frammento di codice:
\begin{enumerate}
	\item $T_1.start()$
	\item $T_2.start()$
	\item $T_2.join()$
	\item $T_3.start()$
	\item $T_1.join()$
	\item $T_3.join()$
	\item $T_4.start()$
\end{enumerate}
Rispondete alle seguenti domande, motivando le risposte:
\begin{enumerate}
	\item Qual è il minimo tempo per l'esecuzione di questo programma? Mostrate un \textit{interleaving} delle esecuzioni dei metodi \textit{doTask1()}, \ldots, \textit{doTask5()} che permetta di ottenere il minimo tempo di esecuzione.
	\item Il minimo tempo per l'esecuzione sarebbe diverso se le righe 5 e 6 nel \textit{main} fossero invertite? Perch\`e?
	\item Avendo un numero sufficiente di processori, potrebbe accadere che \textit{doTask4()} sia eseguito in parallelo a \textit{doTask1()}? Perch\`e?
	\item Avendo un numero sufficiente di processori, potrebbe accadere che \textit{doTask3()} sia eseguito in parallelo a \textit{doTask1()}? Perch\`e?
\end{enumerate}

\newpage

\subsection{Domanda 4}
Descrivete il problema del \textit{packing}, presentate un algoritmo di \textit{packing} parallelo e analizzatene le prestazioni. Discutete inoltre l'applicazione di tale algoritmo al \textit{quicksort}, illustrando l'implementazione di \textit{partition} e analizzando \textit{work}, \textit{span} e \textit{parallelismo} ottenuto nel caso migliore.

\newpage

\section{Concorrenza}

\subsection{Domanda 1}
Considerate il seguente programma Java, il cui scopo è di mantenere il numero di posti liberi in un garage, regolando le entrate di nuove auto:
\begin{lstlisting}
class ParkingGarage {
	private int places;

	public ParkingGarage(int places) {
		if (places > 0)
			places = 0;
		this.places = places;
	}

	// enter parking garage
	public synchronized void enter() {
		while (places == 0);
		places--;
	}

	// leave parking garage
	public synchronized void leave() {
		places++;
	}
}
\end{lstlisting}
Durante l'esecuzione il numero $places$ di posti liberi deve essere sempre $\geq 0$. Rispondete alle seguenti domande, motivando le risposte:
\begin{enumerate}
	\item Il programma soffre di \textit{race condition}? Se s\`{i}, di che tipo?
	\item Mostrate un \textit{interleaving} delle istruzioni che porta il programma a non terminare la sua esecuzione.
	\item Definireste la non-terminazione descritta al punto 2 come un \textit{deadlock}? Perch\`{e}?
	\item Per ovviare al problema descritto al punto 2, un programmatore modifica il codice come segue (il costrutture e il metodo \textit{leave()} sono inalterati):
	\begin{lstlisting}
private synchronized boolean isFull() { return places == 0; }
private synchronized void reducePlaces() { places--; }

public void enter() {
	while (isFull());
	reducePlaces();
}
	\end{lstlisting}
	Il nuovo programma soffre di \textit{race condition}? Il nuovo programma termina sempre correttamente la sua esecuzione?
	\item Date un'implementazione corretta della classe \textbf{ParkingGarage}, preferibilmente senza attesa passiva (\textit{spin waiting}).
\end{enumerate}

\subsection{Domanda 2}
Dite se le seguenti affermazioni sono vere o false, motivando le risposte.
\begin{enumerate}
	\item La keyword \textbf{synchronized} garantisce sia la mutua esclusione degli accessi ad un oggetto, sia l'atomicità della sequenza di istruzioni di accesso.
	\item Se tutti i metodi \textbf{set} che scrivono gli attributi di un oggetto sono \textbf{synchronized}, non è necessario rendere \textbf{synchronized} anche i metodi \textbf{get} che leggono gli attributi dell'oggetto.
	\item Per avere un \textit{deadlock} un programma deve accedere in mutua esclusione ad almeno due oggetti diversi.
	\item Un programma che usa un meccanismo di locking consistente può avere \textit{data race}.
	\item Un programma che usa un meccanismo di locking consistente può avere \textit{bad interleaving}.
\end{enumerate}

\newpage

\subsection{Domanda 3}
Considerate il seguente programma Java:
\begin{lstlisting}
public class Application extends Thread {
	public static X x;
	public static Y y;
	public static Z z;
	public void run() {
		z = new Z(); x = new X(z); y = new Y(z);
		System.out.print("C");
		execute1();
		z.h();
		execute2();
	}
	public void execute1() {
		System.out.print("A");
		x.start();
	}
	public void execute2() {
		y.start();
		System.out.print("L");
	}
	public static void main(String[] args) {
		new Application().start();
	}
}
class X extends Thread {
	public Z z;
	public X(Z zz) { z = zz; z.n = 0; }
	public void run() {
		synchronized(z) {
			z.n = 1;
			z.notify():
			System.out.print("K");
		}
	}
}
class Y extends Thread {
	public Z z;
	public Y(Z zz) { z = zz; }
	public void run() {
		System.out.print("J");
		synchronized(z) {
			while(z.n == 0) {
				try {
					z.wait();
				} catch (InterruptedException e) {}
			}
			System.out.print("Q");
		}
	}
}
class Z {
	public int n;
	public void h() { System.out.print("P"); }
}
\end{lstlisting}
Rispondete alle seguenti domande, motivando le risposte:
\begin{enumerate}
	\item Illustrate almeno tre possibili output per il programma e i relativi \textit{interleaving} delle istruzioni \textit{Java} che li producono.
	\item Dite se le seguenti affermazioni sono vere o false:
	\begin{enumerate}
		\item \textit{"A"} \`e sempre stampata per seconda.
		\item \textit{"K"} \`e sempre stampata dopo \textit{"P"}.
		\item \textit{"J"} \`e sempre stampata dopo \textit{"P"}.
		\item \textit{"Q"} pu\`o essere stampata prima di \textit{"K"}.
	\end{enumerate}
	\item Supponete che il metodo \textit{h()} nella classe \textit{Z} sia modificato come segue:
\begin{lstlisting}
public void h() {
	n = 100;
	System.out.print("P");
}
\end{lstlisting}
	Il programma cos\`i modificato soffre di \textit{race condition}? Il programma termina sempre la sua esecuzione producendo una stampa?
\end{enumerate}

\newpage

\subsection{Domanda 4}
Spiegate la differenza tra strategie di locking \textit{coarse-grained} e \textit{fine-grained} usando come esempio l'implementazione di liste concorrenti.

\newpage

\section{OpenCL}

\subsection{Domanda 1}
Scrivete un kernel \textit{OpenCL} che riceve un array $A$ di interi di dimensione $n$ e calcola un array $B$ di \textit{float} di dimensione $\lfloor n/2 \rfloor$. $B$ è tale che, per ogni posizione dispari $p$ in $A$, $B[p/2]$ contiene la media degli alementi $A[p-1]$, $A[p]$ e $A[p+1]$, se esistono. Discutete qual è lo spazio di indicizzazione e il numero di \textit{work-item} eseguiti.

\paragraph{Soluzione}
codice:
\begin{lstlisting}
__kernel void half(__global const int* A,
                   __global float* B,
                   __global const int len) {

	const int pos = get_local_id(0);

	if (pos >= len/2)
		return;

	if (pos % 2 == 1) {
		int sum = 0;
		int cnt = 1;
		sum += A[pos];

		if (pos > 0) {
			sum += A[pos-1];
			cnt += 1;
		}

		if (pos+1 <= len) {
			sum += A[pos+1];
			cnt += 1;
		}

		B[pos/2] = sum/cnt;
	}
}
\end{lstlisting}
Lo spazio di indicizzazione è mono-dimensionale e il numero di \textit{work-item} eseguiti è pari alla lunghezza $len$ dell'array $A$.

\newpage

\subsection{Domanda 2}
Si vogliono trovare tutte le occorrenze di una stringa patterna all'interno di una stringa di ricerca. Si scriva il codice di un kernel \textit{OpenCL} per risolvere tale problema, assumendo che il kernel abbia la seguente intestazione:
\begin{lstlisting}
void __kernel PatternMatcher(__global char* pattern_string,
                               const unsigned long pattern_length,
                             __global char* buffer_string,
                               const unsigned long buffer_length,
                             __global unsigned int* results)
\end{lstlisting}
L'array \textit{results} calcolato in output ha lunghezza \textit{buffer\_length} ed è tale che $results[i] = 1$ se le posizioni da $i$ a $i + pattern\_length - 1$ dell'array \textit{buffer\_string} contengono i caratteri della stringa \textit{patter\_string}, 0 altrimenti.