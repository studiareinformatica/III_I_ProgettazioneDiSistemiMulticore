I processori multicore, che integrano più di una CPU su un unico chip, sono sempre più diffusi.\\
In teoria, consentono di aumentare la potenza di elaborazione disponibile nella macchina aggirando le limitazioni tecnologiche (velocità di clock, dissipazione termica). In pratica, un programma seriale convenzionale, non potendo utilizzare più di una CPU, non può ottenere i vantaggi prestazionali promessi.\\
Scopo di questo corso è di illustrare le potenzialità e i limiti dell'approccio, quali tecniche adoperare per rendere i codici più efficienti a parità di frequenza di clock, ed introdurre alcuni modelli di programmazione che permettono di suddividere le elaborazioni di un unico programma su più CPU, senza dover ricorrere esplicitamente a processi intercomunicanti o ai \textit{thread}.

\section{La Prima Legge di Moore}
\textit{\textquotedblleft La complessità di un microcircuito, misurata ad esempio tramite il numero di transistori per chip, raddoppia ogni 18 mesi. \textquotedblright}\\\\
La prima legge di Moore è tratta da un'osservazione empirica di Gordon Moore, cofondatore di Intel con Robert Noyce: nel 1965, Gordon Moore, che all'epoca era a capo del settore R\&D della Fairchild Semiconductor e tre anni dopo fondò la Intel, scrisse infatti un articolo su una rivista specializzata nel quale illustrava come nel periodo 1959-1965 il numero di componenti elettronici (ad esempio i transistor) che formano un chip fosse raddoppiato ogni anno. \\
Nel 1975 questa previsione si rivelò corretta e prima della fine del decennio i tempi si allungarono a due anni, periodo che rimarrà valido per tutti gli anni ottanta. La legge, che verrà estesa per tutti gli anni novanta e resterà valida fino ai nostri giorni, viene riformulata alla fine degli anni ottanta ed elaborata nella sua forma definitiva, ovvero che il numero di transistori nei processori raddoppia ogni 18 mesi. Questa legge è diventata il metro e l'obiettivo di tutte le aziende che operano nel settore, come Intel e AMD.